\documentclass[12pt,a4paper]{article}

\usepackage[dutch]{babel}
\usepackage[utf8]{inputenc}

\title{Requirements Engineering: League of Legends eSports}
\author{Bellemans Jonah \and Caerts Stijn \and Dimova Yana \and Lambrichts Midas \and Verhoest Elise \and Vos Pieter}
\date{\today}


\begin{document}
	\maketitle
	\newpage
	\tableofcontents
	\newpage
	\section{Motivatie sportkeuze}
		We opteerden om e-sports, meer bepaald de "League of Legends World Championships (LoL Worlds)", te kiezen als sporttak voor dit project. Wij kozen hiervoor omdat deze sporttak origineel is, en een zeer duidelijke afbakening van de spel- en wedstrijdreglementen voorziet. Aangezien de meerderheid van de groepsleden vanuit een computer science-gerelateerde achtergrond komt, is de e-sport goed door de leden gekend, en is iedereen gemotiveerd om hieraan te werken.
	\section{Beschrijving Competitie}
		\subsection{Plaatsingsmatches}
			Teams plaatsen zich voor de competitie door middel van plaatsingswedstrijden die vroeger in het seizoen plaatsvinden. Voor elke regio (West-Europa, Noord-Amerika, Korea,\dots) worden een aantal "Seeds" vrijgehouden, die worden ingevuld door de beste ploegen van die regionale competities. De absolute topploegen van de wereld, zoals winnaars van eerdere jaren, kunnen via een "wildcard" rechtstreeks uitgenodigd worden voor deelname door het organisatorisch comité.
		\subsection{Wedstrijdverloop}
			Elke wedstrijd van het spel League of Legends ("LoL") wordt gespeeld door twee teams van elk vijf spelers. De teamleden besturen elk een individueel personage in de game, die men "Champions" noemt. Na het selecteren van de champions voor elke speler, worden de teams in de spelwereld ("Summoner's Rift") gezet. Het doel van het spel is om de "Nexus" van het andere team te vernietigen, terwijl men de eigen Nexus verdedigt. Het spel blijft lopen totdat er de nexus van één van beide teams vernietigd is. Het is dus onmogelijk om een match te beeïndigen zonder dat er een éénduidige winnaar bepaald is.
		\subsection{Competitieverloop}
			\subsubsection{Eerste competitieronde: Group Stage}
				De wereldkampioenschappen beginnen met 16 verschillende teams, die opgedeeld worden in groepen. Binnen de groep spelen de teams telkens twee aan twee tegen elkaar. De "matchups" worden bepaald met een round-robin systeem, waarbij elk team tweemaal niet-opeenvolgend tegen elkaar speelt (eenmaal aan elke zijde van Summoner's Rift).
				\paragraph{Tiebreaker}
				Als twee teams dezelfde score behalen, zal de zogenaamde "head-to-head" score (de score die door de teams behaald werd tijdens hun onderlinge wedstrijden) gebruikt worden om een winnaar te bepalen. Als ook hieruit geen winnaar besloten kan worden, zal een enkele tiebreak match gespeeld worden.
			\subsubsection{Tweede competitieronde: Quarterfinals}
				Tijdens de kwartfinales ("Quarterfinals") blijven slechts acht seeds over, die ingenomen worden door de hoogst scorende teams tijdens de Group Stage. De teams die \#1 werden binnen hun groep, spelen tegen de teams die \#2 werden in de andere groep. Welk team tegen wie speelt, wordt dan willekeurig bepaald. Eens er een matchup bepaald is, ligt daarmee ook de "samenhangende" matchup vast. Als de willekeurige trekking dus bepaalt dat Team 1 uit groep A speelt tegen Team 2 uit groep D, dan zal ook Team 2 uit groep A spelen tegen Team 1 uit groep D. Nadat alle matchups vastliggen, spelen de gekozen teams telkens een "best-of-five" match, waaruit de winnaar doorstroomt naar de volgende ronde.
			\subsubsection{Derde competitieronde: Semifinals}
				In de halve finale ("Semifinals") spelen de winnende teams uit de kwartfinales opnieuw een "best-of-five" match tegen elkaar. De twee overblijvende winnende teams gaan vervolgens naar de finale.
			\subsubsection{Finale}
				De finale wordt gespeeld door de twee overgebleven teams. Ook hier wordt de uiteindelijke winnaar bepaald door een "best-of-five" match. Het team dat op het einde van deze vijf games het meeste gewonnen heeft, is de uiteindelijke winnaar van het toernooi.
	\section{Conclusie}
\end{document}