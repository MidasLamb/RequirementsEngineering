\documentclass[12pt,a4paper]{article}

\usepackage[dutch]{babel}
\usepackage[utf8]{inputenc}

\title{Requirements Engineering: League of Legends eSports}
\author{Bellemans Jonah \and Caerts Stijn \and Dimova Yana \and Lambrichts Midas \and Verhoest Elise \and Vos Pieter}
\date{\today}


\begin{document}
	\maketitle
	\newpage
	\tableofcontents
	\newpage
	\section{Motivatie sportkeuze}
		We opteerden om e-sports, meer bepaald de "League of Legends World Championships (LoL Worlds)", te kiezen als sporttak voor dit project. Wij kozen hiervoor omdat deze sporttak origineel is, en een zeer duidelijke afbakening van de spel- en wedstrijdreglementen voorziet. Aangezien de meerderheid van de groepsleden vanuit een computer science-gerelateerde achtergrond komt, is de e-sport goed door de leden gekend, en is iedereen gemotiveerd om hieraan te werken.
	\section{Beschrijving Competitie}
		\subsection{Plaatsingsmatches}
			Teams plaatsen zich voor de competitie door middel van plaatsingswedstrijden die vroeger in het seizoen plaatsvinden. Voor elke regio (West-Europa, Noord-Amerika, Korea,\dots) worden een aantal "Seeds" vrijgehouden, die worden ingevuld door de beste ploegen van die regionale competities. De absolute topploegen van de wereld, zoals winnaars van eerdere jaren, kunnen via een "wildcard" rechtstreeks uitgenodigd worden voor deelname door het organisatorisch comité.
		\subsection{Wedstrijdverloop}
			Elke wedstrijd van het spel League of Legends ("LoL") wordt gespeeld door twee teams van elk vijf spelers. De teamleden besturen elk een individueel personage in de game, die men "Champions" noemt. Na het selecteren van de champions voor elke speler, worden de teams in de spelwereld ("Summoners' Rift") gezet. Het doel van het spel is om de "Nexus" van het andere team te vernietigen, terwijl men de eigen Nexus verdedigt.
		\subsection{Competitieverloop}
			\subsubsection{Eerste competitieronde: "Group Stage"}
				De wereldkampioenschappen beginnen met 18 verschillende teams, die telkens twee aan twee tegen elkaar spelen. 
			\subsubsection{Tweede competitieronde: "Quarterfinals"}
			\subsubsection{Derde competitieronde: "Semifinals"}
			\subsubsection{Finale}
	\section{Conclusie}
\end{document}